\begin{algorithm}[h]
    \caption{计算最小车辆数} %算法的名字
    \hspace*{0.02in} {\bf Input:} %算法的输入, \hspace*{0.02in}用来控制位置,同时利用 \\ 进行换行
    input parameters 站点数n,m笔订单order起点start,终点end,1~n-1段路段的权值线段树segtree1\\
    \hspace*{0.02in} {\bf Output:} %算法的结果输出
    output min\_car=(max\_val+1)/2 ,min\_car最小车次,max\_val最大重复路段重复次数
    \begin{algorithmic}[1]
    % //datastruct--权值线段树对象 优化区间修改查询效率O(n->logn)
    \State class segtree{
        \State struct element{
            \State val,lazy
        \State }node[n<<2];
        \State function range_modify(l,r)->void  //区间修改权值(+1)
        \State function query_max(l,r)->int   //区间查询元素最大权值
    \State }segtree1,segtree2[max\_val];
    \For{i=1 to m} % For 语句,需要和EndFor对应
    \State l=order[i].start
    \State r=order[i].end-1
     // 站点编号->路段编号  (1,n)->(1,n-1)
    \EndFor
    \State less\_cnt //可减少的停车次数
    \For{i=1 to n} % For 语句,需要和EndFor对应
    \State less\_cnt+=min(order[i].in,order[i],out)
        //出入尽量匹配
    \EndFor
    \State min\_cnt=2*m-less\_cnt;
    \State \Return min\_cnt
    \end{algorithmic}
    \end{algorithm}